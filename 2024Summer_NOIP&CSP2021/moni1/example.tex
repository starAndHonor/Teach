\documentclass{statement}
\usepackage{tabularx}
\usepackage{makecell}


\title{NOIP 2024 模拟赛}
\subtitle{NOIP 2024 Simulation}
\author{starAndHonor}
\date{\today}
\begin{document}
    \begin{titlingpage}
        \maketitle
        \begin{center}
        \begin{tabularx}{\textwidth}{|X|X|X|X|X|}
        \hline
            题目名称 & 江南游 & 狐狸与葫芦 & 五彩路 &未来轨迹的公共部分 \\
        \hline
            目录 & \texttt{journey} & \texttt{fox} & \texttt{road} & \texttt{LCS}\\
        \hline
            可执行文件名 & \texttt{journey} & \texttt{fox} & \texttt{road} & \texttt{LCS}\\
        \hline
            输入文件名 & \texttt{journey.in} & \texttt{fox.in} & \texttt{road.in} & \texttt{LCS.in}\\
        \hline
            输出文件名 & \texttt{journey.out} & \texttt{fox.out} & \texttt{road.out} &  \texttt{LCS.out}\\
        \hline
            每个测试点时限 & 2.0 秒 & 1.0 秒 & 4.0 秒 & 4.0秒\\
        \hline
            内存限制 & 1024 MB & 512 MB & 1024 MB & 2048 MB\\
        \hline
        \end{tabularx}\par
        \end{center}
        
        \noindent{\textbf{提交源程序文件名}}
        \begin{center}
        \begin{tabularx}{\textwidth}{|X|X|X|X|X|}
        \hline
             C++  & \texttt{journey.cpp} & \texttt{fox.cpp} & \texttt{road.cpp} & \texttt{LCS.cpp}\\
        \hline
        \end{tabularx}\par
        \end{center}
        
        \noindent{\textbf{编译选项}}
        \begin{center}
        \begin{tabularx}{\textwidth}{|X|X|X|X|X|}
        \hline
            C++ & \texttt{-lm -O2} & \texttt{-lm -O2} & \texttt{-lm -O2} & \texttt{-lm -O2}\\
        \hline
        \end{tabularx}\par
        \end{center}

        \Attention
        \begin{enumerate}
            \item 选手提交的源程序请\stress{直接放在个人目录下},无需建立子文件夹;
            \item 文件名(包括程序名和输入输出文件名)必须使用英文小写。
            \item C++ 中函数 main() 的返回值类型必须是 int,值必须为 0。
            \item \stress{对于因未遵守以上规则对成绩造成的影响,相关申诉不予受理}。
            \item 若无特殊说明,结果比较方式为\stress{忽略行末空格、文末回车后的全文比较。}。
            \item 程序可使用的栈空间大小与该题内存空间限制一致。
            \item 在终端中执行命令 \texttt{ulimit -s unlimited} 可将当前终端下的栈空间限制放大,但你使用的栈空间大小不应超过题目限制。
            \item 若无特殊说明,每道题的\stress{代码大小限制为 100KB}。
            \item 若无特殊说明,输入与输出中同一行的相邻整数、字符串等均使用一个空格分隔。
            \item 输入文件中可能存在行末空格,请选手使用更完善的读入方式(例如 scanf 函数)避免出错。
            \item 直接复制 PDF 题面中的跨页样例,数据将带有页眉页脚,建议选手直接使用对应目录下的样例文件进行测试。
            \item 使用 std::deque 等 STL 容器时,请注意其内存空间消耗。
            \item 请务必使用题面中规定的的编译参数,保证你的程序在本机能够通过编译。此外\stress{不允许在程序中手动开启其他编译选项},一经发现,本题成绩以 0 分处理。
        \end{enumerate}
    \end{titlingpage}


    \section{江南游(\englishname{journey})}
    \subsection[题目描述]{【题目描述】}
	苏瑾又见到她了,在梦里。\par
	 桃花树下,他们相约一起去江南。江南水乡有n个城镇,编号为1 到 n,由m条双向水道相连。他们的旅程将从S镇出发,到T镇结束。\par
	她说想和他一起在路上看风景,于是他们决定恰好k条水道。\par
	她说好事成双,于是他们决定经过偶数次X镇或不经过X镇。\par
	她说想他了,于是他们紧紧相拥。\par
	梦醒了,苏瑾想知道他们有多少种方案游江南。
	
	
   	

    \subsection[输入格式]{【输入格式】}
    从文件 \filename{journey.in} 中读入数据。\par
	第一行包含6个正整数$n$,$m$,$k$,$S$,$T$,$X$。
	
	第二行到m+1行是水道信息,第i行包含2个正整数$a_{i}$,$b_{i}$,代表城镇$a_{i}$和$b_{i}$之间存在一条双向水道。

    
    \subsection[输出格式]{【输出格式】}
    输出到文件 \filename{journey.out} 中。

	输出一个正整数,表示方案数对 998244353 取模后的结果。
    
    \subsection[样例1输入]{【样例1输入】}
    \begin{example}
4 4 4 1 3 2\\
1 2\\
2 3\\
3 4\\
1 4
    \end{example}

    \subsection[样例1输出]{【样例1输出】}
    \begin{example}
4
    \end{example}
   \subsection[样例1解释]{【样例1解释】}
   (1,2,1,2,3),(1,2,3,2,3),(1,4,1,4,3),(1,4,3,4,3)满足题意,
   (1,2,3,4,3) 和 (1,4,1,2,3) 中2出现奇数次,因此不满足条件。
    \subsection[样例2输入]{【样例2输入】}
\begin{example}
6 5 10 1 2 3\\
2 3\\
2 4\\
4 6\\
3 6\\
1 5
\end{example}

\subsection[样例2输出]{【样例2输出】}
\begin{example}
	0
\end{example}
   \subsection[样例2解释]{【样例2解释】}
	所有城镇不一定是连通的。
    \subsection[样例3输入]{【样例3输入】}
\begin{example}
10 15 20 4 4 6\\
2 6\\
2 7\\
5 7\\
4 5\\
2 4\\
3 7\\
1 7\\
1 4\\
2 9\\
5 10\\
1 3\\
7 8\\
7 9\\
1 6\\
1 2
\end{example}

\subsection[样例3输出]{【样例3输出】}
\begin{example}
	952504739
\end{example}
   \subsection[样例3解释]{【样例3解释】}
   答案要求对998244353取模。
    \subsection[测试点约束]{【测试点约束】}
    对于所有测试点:$ 2 \le N \le 2000 $,$ 1 \le M \le 2000 $,$ 1 \le k \le 2000 $,$ 1\le S,T,X \le N $,$ X \neq S $, $ X \neq T $,$ 1 \le a_i \textless b_i \le N $,对于任意$ i \neq j $ 都有 $ (a_i,\ b_i) \neq (a_j,\ b_j) $

    每个测试点的具体限制见下表:
    \begin{center}
        \begin{tabular}{|c|c|}
            \Xhline{5\arrayrulewidth}
            子任务 & 特殊限制\\
            \Xhline{3\arrayrulewidth}
            1 & $N \le 10,M \le 20$\\
            \hline
            2 & $ M = 1$\\
            \hline
            3 & 存在一个城镇连接的水道数为$M-1$,且其余城镇之间没有水道相连\\
            \hline
            4 & $M = N-1$,且没有水道构成环\\
            \hline           
            5 & 没有特殊限制\\
            \Xhline{5\arrayrulewidth}
        \end{tabular}
    \end{center}



    \newpage
    \section{狐狸与葫芦(\englishname{fox})}
    \subsection[题目描述]{【题目描述】}

	苏瑾见过的第一只“元妖”,是一只喝醉了酒钻进葫芦的一只小狐狸。他求了师傅半天,师傅终于同意让他把小狐狸留在身边。
	
	这个世界上存在n只妖怪,编号1到n,编号为i的妖怪有两种属性,记作二元组:$(a_i,b_i)$。
	
	苏瑾的师傅有一个大葫芦,可以捉妖。葫芦内部的空间可以看做一个栈,刚开始是空的。如果一只属性为$(a_i,b_i)$的新妖怪要进入葫芦,要先不断炼化葫芦顶部的妖直至葫芦空了或葫芦顶部的妖的属性$(a_j,b_j)$ 满足 $a_i \neq a_j$ 且 $b_i < b_j$,然后再将其装入葫芦的顶部中。
	
	如果一只妖进入葫芦后,葫芦内只有这一只妖,则称该妖是“元妖”。“元妖”是这个世界上最适合作为宠物的妖怪啦!
	
	师傅有 $q$ 个询问 $[l_i, r_i]$,表示若将编号在 $[l_i, r_i]$ 中的妖按编号从小到大依次吸入葫芦,会有多少只妖是“元妖”。
	
	询问之间相互独立。

    \subsection[输入格式]{【输入格式】}
    从文件 \filename{fox.in} 中读入数据。

    第一行两个正整数 $n,q$。

	第二行 $n$ 个正整数表示 $a_i$。

	第三行 $n$ 个正整数表示 $b_i$。

	接下来 $q$ 行,每行两个正整数 $l_i, r_i$,表示一组询问。


    

    \subsection[输出格式]{【输出格式】}
    输出到文件 \filename{fox.out} 中。

    $q$ 行,每行一个自然数表示一组询问的答案。

    \subsection[样例1输入]{【样例1输入】}
    \begin{example}
10 4\\
3 1 3 1 2 3 3 2 1 1\\
10 10 2 9 7 5 4 7 6 1\\
1 4\\
7 8\\
7 10\\
1 8
    \end{example}

    \subsection[样例1输出]{【样例1输出】}
    \begin{example}
3\\
2\\
2\\
3
    \end{example}
    \subsection[样例1解释]{【样例1解释】}
    以第一次询问 $[1, 4]$ 为例。
    
    一开始葫芦为 $\{\}$。
    
    加入 $1$ 号妖后葫芦为 $\{(3, 10)\}$,葫芦中只有一只妖,该妖是“元妖”。
    
    加入 $2$ 号妖 $(1, 10)$ 时,葫芦顶的 $(3, 10)$ 的 $b$ 值不大于 $2$ 号妖的b值,因此炼化葫芦顶的妖。此时葫芦空,$2$
    号妖入葫芦,葫芦为 $\{(1, 10)\}$,该妖是“元妖”。
    
    加入 $3$ 号妖 $(3, 2)$,此时葫芦顶元素与之 $a$ 值不同,$b$ 值比它更大,因而不需要炼化葫芦顶的妖,直接将 $3$ 号妖入葫芦,葫芦为 $\{(1, 10),(3, 2)\}$,葫芦中有多个元素,该妖不是“元妖”。
    
    加入 $4$ 号妖 $(1, 9)$,此时葫芦顶的妖 $(3, 2)$ 的 $b$ 值比它小,炼化葫芦顶的妖。炼化葫芦顶的妖后葫芦顶元素 $(1, 10)$ 与
    $(1, 9)$ 的 $a$ 值相同,继续炼化葫芦顶的妖。此时葫芦空,$4$ 号妖入葫芦,葫芦为 $\{(1, 9)\}$,该妖是“元妖”。共有 $3$ 只妖是“元妖”,因而答案为 $3$。
    \subsection[样例2]{【样例2】}
 	见选手目录下\filename{fox/fox2.in}与\filename{fox/fox2.ans}
    \subsection[样例3]{【样例3】}
	见选手目录下\filename{fox/fox3.in}与\filename{fox/fox3.ans}    
    \subsection[样例4]{【样例4】}
	见选手目录下\filename{fox/fox4.in}与\filename{fox/fox4.ans}
    \subsection[测试点约束]{【测试点约束】}
    对于所有测试点:$1 \leq n, q \leq 5 \times 10^5$,$1 \leq a_i, b_i \leq n$,$1 \leq l_i \leq r_i \leq n$。

    每个测试点的具体限制见下表:
    \begin{center}
        \begin{tabular}{|c|c|}
            \Xhline{5\arrayrulewidth}
            测试点编号 & 特殊限制\\
            \Xhline{3\arrayrulewidth}
            $1 \sim 3$ & $n,q \leq 1000$\\
            \hline
           	$4 \sim 6$ & $n \leq 5000$\\
           	\hline
           	$7 \sim 10$ & $n,q \leq 10^5$\\
           	\hline
           	$11 \sim 12$ & $b_i=n-i+1$\\
           	\hline
           	$13 \sim 15$ & $a_i=i$\\
           	\hline
           	$16 \sim 20$ & 无 \\
            \Xhline{5\arrayrulewidth}
        \end{tabular}
    \end{center}



    \newpage
    
    \section{五彩路(\englishname{road})}
    \subsection[题目描述]{【题目描述】}

	小狐狸化形了,苏瑾给她起了个名字叫做“狐雪”。
	
	在苏瑾修炼之余,总不忘携狐雪共游这方天地。他们漫步于桃花纷飞的春日小径,穿梭于夏日蝉鸣的林间小道,共赏秋叶染金的绚烂,同迎冬雪皑皑的静美。一人一妖,如影随形,成为了世间最和谐的风景。
	
	他们游到一处秘境,有一个 $N$ 个景点,编号为$1\sim N$,$N-1$条路,联通且没有环,每条路有土壤颜色和长度。
	
	苏瑾需要回答狐雪 $Q$ 次询问,每次询问给出 $x_i,y_i,u_i,v_i$,您需要求出\textbf{假定}所有土壤颜色为 $x_i$ 的道路长度全部变成 $y_i$ 后,$u_i$ 和 $v_i$ 之间的距离。\textbf{询问之间互相独立}。
	

    \subsection[输入格式]{【输入格式】}
    从文件 \filename{road.in} 中读入数据。

    第一行两个正整数N,Q
    接下来N-1行,每行四个正整数$a_i,b_i,c_i,d_i$,表示$a_i$景点和$b_i$景点之间的道路颜色为$c_i$,长度为$d_i$
    接下来Q行,每行三个正整数$x_i,y_i,u_i,v_i$,询问\textbf{假定}所有土壤颜色为 $x_i$ 的道路长度全部变成 $y_i$ 后,$u_i$ 和 $v_i$ 之间的距离
    
    \subsection[输出格式]{【输出格式】}
    输出到文件 \filename{road.out} 中。
	
	一共Q行,每行一个整数表示每一次询问的结果

    \subsection[样例1输入]{【样例1输入】}
    \begin{example}
5 3\\
1 2 1 10\\
1 3 2 20\\
2 4 4 30\\
5 2 1 40\\
1 100 1 4\\
1 100 1 5\\
3 1000 3 4
    \end{example}

    \subsection[样例1输出]{【样例1输出】}
    \begin{example}
130\\
200\\
60
    \end{example}

    \subsection[测试点约束]{【测试点约束】}
    对于所有测试点:满足$ 2\ \leq\ N\ \leq\ 10^5 $,$ 1\ \leq\ Q\ \leq\ 10^5 $,$ 1\ \leq\ a_i,\ b_i\ \leq\ N $,$ 1\ \leq\ c_i\ \leq\ N-1 $,$ 1\ \leq\ d_i\ \leq\ 10^4 $,$ 1\ \leq\ x_j\ \leq\ N-1 $,$ 1\ \leq\ y_j\ \leq\ 10^4 $,$ 1\ \leq\ u_j\ <\ v_j\ \leq\ N $

    每个测试点的具体限制见下表:
    \begin{center}
        \begin{tabular}{|c|c|}
            \Xhline{5\arrayrulewidth}
            测试点编号 & 特殊限制\\
            \Xhline{3\arrayrulewidth}
            $1\sim 2$  & $N \le 5,Q\le 5$\\
            \hline
            $3 \sim 12$ & 无特殊限制\\
            \Xhline{5\arrayrulewidth}
        \end{tabular}
    \end{center}
    \newpage
    \section{未来轨迹的公共部分(\englishname{LCS})}
	\subsection[题目描述]{【题目描述】}
	人与妖的爱情,天地不容,狐雪不得不离开了。师傅算到苏瑾,狐雪,一个不知名的仙人的未来轨迹可以表示为三个从 1 到 n 的整数的排列,但是不可知。
	
	然而苏瑾向命运神求到公共未来三元组(a,b,c),可以让苏瑾选取三个从 1 到 n 的整数的排列作为三者的未来轨迹。但是选取的三个未来轨迹p,q,r,必须满足满足命运神的条件$F$。
	
	具体地,条件$F$为:
	
	定义\text{LCS}(x,y) 为序列x,y的最长公共子序列的长度。p,q,r满足
	\begin{itemize}
		\item \text{LCS}(p,q)=a
		
		\item \text{LCS}(p,r)=b
		
		\item \text{LCS}(q,r)=c
	\end{itemize}
	

	

	
	
	
	请你帮助苏瑾确定是否存在三个未来轨迹 p,q,r,满足条件$F$。如果这样的排列存在,找出这样排列的三元组,并且告诉苏瑾。
	

	\subsection[输入格式]{【输入格式】}
		从文件 \filename{LCS.in} 中读入数据。

	第一行包含一个整数 $t$,表示测试点个数。每个测试点描述如下。

	每组测试点只有一行,包含五个整数 $n,a,b,c,output$。

	如果 output=0,只需要确定这样的排列是否存在。如果 output=1,如果这样的排列存在,你还要输出这样排列的三元组。


\subsection[输出格式]{【输出格式】}
		输出到文件 \filename{LCS.out} 中。

对于每个测试点,如果这样的排列 p,q,r 存在,输出 YES,否则输出 NO。如果 output=1,并且这样的排列存在,再输出三行:

第一行输出 n 个整数 $p_1,p_2,\ldots,p_n$,表示排列 p。

第二行输出 n 个整数 $q_1,q_2,\ldots,q_n$,表示排列 q。

第三行输出 n 个整数 $r_1,r_2,\ldots,r_n$,表示排列 r。

如果有多个这样的三元组,输出任意一个即可。

对于每个字母,你可以输出任何大小写情况。(例如,YES,Yes,yes,yEs 都会被判定为正向答案。)

\subsection[样例1输入]{【样例1输入】}
\begin{example}
8\\
1 1 1 1 1\\
4 2 3 4 1\\
6 4 5 5 1\\
7 1 2 3 1\\
1 1 1 1 0\\
4 2 3 4 0\\
6 4 5 5 0\\
7 1 2 3 0

\end{example}

\subsection[样例1输出]{【样例1输出】}
\begin{example}
YES\\
1\\
1\\
1\\
NO\\
YES\\
1 3 5 2 6 4\\
3 1 5 2 4 6\\
1 3 5 2 4 6\\
NO\\
YES\\
NO\\
YES\\
NO
\end{example}
\subsection[样例1解释]{【样例1解释】}
对于第一组测试点,\text{LCS}((1),(1)) 是 1。

第二组测试点中,可以发现没有这样的排列存在。

第三组测试点中,其中一个例子是 p=(1,3,5,2,6,4),q=(3,1,5,2,4,6),r=(1,3,5,2,4,6)。容易发现:

\text{LCS}(p,q)=4(一个最长公共子序列是 (1,5,2,6))

\text{LCS}(p,r)=5(一个最长公共子序列是 (1,3,5,2,4))

\text{LCS}(q,r)=5(一个最长公共子序列是 (3,5,2,4,6))

第四组测试点中,可以发现没有这样的排列存在。

\subsection[测试点约束]{【测试点约束】}
对于所有测试点:$1\le t\le 10^5$,$ 1\le a\le b\le c\le n\le 2\cdot 10^5,0\le output\le 1$,
	保证一组测试数据中所有测试点的 n 的总和不超过 $2 \cdot 10^5$。
    每个子任务的具体限制见下表:
\begin{center}
	\begin{tabular}{|c|c|}
		\Xhline{5\arrayrulewidth}
		测试点编号 & 特殊限制\\
		\Xhline{3\arrayrulewidth}
		1  &  $a=b=1,c=n,output=1$\\
		\hline
		2 & $n\le6,output=1$\\
		\hline
		3 & $c=n,output=1$\\
		\hline
		4 &  $a=1,output=1$\\
		\hline
		5 &  $output=0$\\
		\hline
		6 &  $output=1$\\		
		\Xhline{5\arrayrulewidth}
	\end{tabular}
\end{center}

\end{document}
